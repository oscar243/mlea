\documentclass[a4paper,10pt,twocolumn]{article}

% --------------------------------------------------------------------
% PAQUETES
% --------------------------------------------------------------------
\usepackage[utf8]{inputenc}
\usepackage[spanish,es-tabla]{babel}
\usepackage[T1]{fontenc}
\usepackage{mathptmx} % Fuente Times New Roman (Estándar científico)
\usepackage{geometry}
\geometry{top=2cm, bottom=2cm, left=1.5cm, right=1.5cm} % Márgenes optimizados

\usepackage{graphicx}
\usepackage{float}
\usepackage{subcaption}
\usepackage[labelfont=bf,font=small]{caption} % Captions más compactos
\usepackage{booktabs}
\usepackage{multirow}
\usepackage{amsmath}
\usepackage{amssymb}
\usepackage{xcolor}
\usepackage{listings}
\usepackage{hyperref}
\usepackage{abstract} % Mejor control del abstract

% Configuración de hipervínculos
\hypersetup{
    colorlinks=true,
    linkcolor=blue,
    citecolor=blue,
    urlcolor=blue
}

\title{\textbf{\Large Informe de Proyecto: Reconocimiento de Actividad Humana (HAR)}}
\author{\textbf{Oscar A. Virguez} \\ \textit{MLEA 2025-II - qwertylab.dev}}
\date{\today}

\begin{document}

\twocolumn[
  \begin{@twocolumnfalse}
    \maketitle
    \begin{abstract}
        \textit{[Escribir aquí el resumen ejecutivo: Problema + Metodología + Resultado Clave + Conclusión de impacto. Debe ser un párrafo conciso que invite a la lectura del resto del documento.]}
        \vspace{0.5cm}
    \end{abstract}
  \end{@twocolumnfalse}
]

\section{Introducción}
El presente documento detalla el flujo de trabajo seguido para realizar una solución al desafío de Reconocimiento de Actividad Humana (HAR), abarcando:

\begin{enumerate}
    \item \textbf{Análisis Exploratorio de Datos (EDA):} Comprensión de la distribución de clases y características de las señales.
    \item \textbf{Preprocesamiento e Ingeniería de Características:} Limpieza de datos, normalización y extracción de atributos relevantes para mejorar el desempeño de los modelos.
    \item \textbf{Modelado Predictivo:} Entrenamiento y evaluación de diversos algoritmos (incluyendo XGBoost y otros enfoques de aprendizaje supervisado).
    \item \textbf{Optimización y Selección:} Ajuste de hiperparámetros y selección del modelo óptimo balanceando precisión (\textit{accuracy}) y eficiencia computacional.
    \item \textbf{Generación de Resultados:} Producción del archivo de predicciones para la competencia.
\end{enumerate}

[Escribir aquí la introducción del problema, contexto del HAR y objetivos del proyecto.]

\section{Metodología y Análisis Exploratorio}

\subsection{Unificación y Carga de Datos}
Se procedió a la carga y unificación de los conjuntos de datos provenientes de tres fuentes: entrenamiento, prueba y validación (Kaggle). Se consolidaron las señales de acelerometría y los metadatos correspondientes en un único estructura de datos para facilitar el preprocesamiento conjunto. Se verificó la consistencia de las dimensiones y se etiquetó cada registro con su fuente de origen.

\subsection{Calidad de los Datos}
Se realizó una auditoría de calidad enfocada en la longitud de las secuencias temporales. Se confirmó que la gran mayoría de los \textit{snippets} cumplen con la longitud estándar de 100 muestras (correspondientes a 5 segundos a 20 Hz). Este hallazgo permite asumir vectores de características de longitud fija para las etapas posteriores de modelado.

Además, el análisis preliminar sugiere una alta correlación entre las métricas de tendencia central (media, mediana, suma) y una redundancia matemática entre la varianza y la desviación estándar.

\subsection{Análisis Exploratorio de Datos (EDA)}

\subsubsection{Balance de Clases}
El análisis de la distribución de clases en el conjunto de entrenamiento revela el balance entre las diferentes actividades. Se utilizó una paleta de colores cualitativa (Set2) para distinguir claramente cada categoría.

\begin{figure}[h]
    \centering
    \includegraphics[width=\linewidth]{../plots/class_distribution.png}
    \caption{Distribución de clases en el conjunto de entrenamiento.}
    \label{fig:class_dist}
\end{figure}

\subsubsection{Visualización de Señales}
Se inspeccionaron visualmente las señales de acelerometría en los tres ejes (X, Y, Z) para muestras representativas de cada actividad. Esto permitió identificar patrones temporales básicos y diferencias de amplitud.

\begin{figure*}[t]
    \centering
    \includegraphics[width=\textwidth]{../plots/signals_examples.png}
    \caption{Ejemplos de señales de acelerómetro por actividad (Small Multiples).}
    \label{fig:signals}
\end{figure*}

\subsubsection{Correlación de Características}
Se calculó la matriz de correlación entre las variables numéricas de los metadatos para identificar redundancias. Se utilizó un mapa de color divergente (\textit{coolwarm}) centrado en cero para visualizar correlaciones positivas y negativas.

\begin{figure}[h]
    \centering
    \includegraphics[width=\linewidth]{../plots/correlation_matrix.png}
    \caption{Matriz de correlación de características (Metadata).}
    \label{fig:corr}
\end{figure}

\subsection{Análisis de Orientación y SVM}
Se investigó la dependencia de las mediciones respecto a la orientación del dispositivo. Los gráficos de dispersión 3D de los promedios por eje mostraron que, incluso para actividades estáticas como \textit{Standing}, existe una variabilidad esférica que sugiere cambios en la posición del sensor (rotación).

\begin{figure*}[t]
    \centering
    \begin{subfigure}[b]{0.48\textwidth}
        \includegraphics[width=\textwidth]{../plots/3d_orientation_distribution.png}
        \caption{Distribución 3D de promedios.}
    \end{subfigure}
    \hfill
    \begin{subfigure}[b]{0.48\textwidth}
        \includegraphics[width=\textwidth]{../plots/3d_standing_gravity.png}
        \caption{Efecto de gravedad en Standing.}
    \end{subfigure}
    \caption{Análisis de orientación y efecto de la gravedad.}
    \label{fig:orientation}
\end{figure*}

El análisis de orientación revela que, aunque los ejes mantienen cierta alineación general, existen variaciones notables entre muestras debido a la rotación del dispositivo. Utilizar la \textbf{Magnitud del Vector de Señal (SVM)} para obtener características invariantes a la posición del sensor puede ayudar a diferenciar mejor las señales que usar las componentes cartesianas. Los resultados sugieren que un sistema de coordenadas esféricas se adapta mejor a la física del movimiento (especialmente bajo la influencia constante de la gravedad), y que la amplitud (magnitud) resulta ser una variable más efectiva e interpretable, permitiendo capturar la intensidad de la actividad sin el ruido introducido por los cambios de dirección.

Debido a esta variabilidad, se decidió utilizar la \textbf{Magnitud del Vector de Señal (SVM)} como característica principal invariante a la rotación:
\begin{equation}
    SVM = \sqrt{x^2 + y^2 + z^2}
\end{equation}

\subsection{Análisis en el Dominio de la Frecuencia}
Para caracterizar mejor las actividades dinámicas frente a las estáticas, se analizaron las propiedades espectrales de la señal SVM.

Se visualizó la "huella digital" de cada actividad superponiendo múltiples señales de SVM. Esto permite observar la varianza intra-clase y patrones periódicos característicos (ej. pasos en \textit{Walking} vs ruido en \textit{Sitting}).

\begin{figure*}[t]
    \centering
    \includegraphics[width=\textwidth]{../plots/svm_fingerprint.png}
    \caption{Huella Digital de Actividades: Superposición de señales SVM.}
    \label{fig:fingerprint}
\end{figure*}

\subsubsection{Densidad Espectral de Potencia (PSD)}
El periodograma de Welch permitió visualizar la distribución de energía en frecuencia para cada actividad.

\begin{figure}[h]
    \centering
    \includegraphics[width=\linewidth]{../plots/psd_average.png}
    \caption{Densidad Espectral de Potencia promedio por actividad.}
    \label{fig:psd}
\end{figure}

\subsubsection{Autocorrelación}
La autocorrelación mide la similitud de una señal consigo misma desplazada en el tiempo (lag). Para actividades periódicas (caminar, correr), la autocorrelación muestra picos claros y repetitivos correspondientes al periodo del movimiento. Por el contrario, para actividades aperiódicas o estáticas, la autocorrelación decae rápidamente sin mostrar picos secundarios significativos.

\begin{figure*}[t]
    \centering
    \includegraphics[width=\textwidth]{../plots/autocorrelation.png}
    \caption{Análisis de Autocorrelación por Actividad.}
    \label{fig:autocorr}
\end{figure*}

\subsubsection{Espectrogramas y Entropía}
Los espectrogramas revelaron la evolución temporal de las frecuencias, mientras que la entropía espectral cuantificó la complejidad de la señal, resultando ser un discriminador potente entre actividades estáticas (alta entropía/ruido) y dinámicas (baja entropía/cíclicas).

El análisis en el dominio de la frecuencia confirma una clara distinción entre actividades estáticas (\textit{Standing}, \textit{Sitting}) y dinámicas. Mientras que actividades como \textit{Jogging}, \textit{Walking} y subir/bajar escaleras presentan firmas espectrales únicas y bien diferenciadas, el principal desafío radica en distinguir entre \textit{Sitting} y \textit{Standing}. Ambas muestran perfiles de frecuencia muy similares y las características calculadas no revelan diferencias significativas a simple vista; sin embargo, una inspección detallada de los espectrogramas sugiere una densidad espectral ligeramente mayor en \textit{Standing}, lo cual podría ser un factor clave para su clasificación.

\begin{figure}[h]
    \centering
    \includegraphics[width=\linewidth]{../plots/spectral_entropy.png}
    \caption{Distribución de la Entropía Espectral por actividad.}
    \label{fig:entropy}
\end{figure}

\section{Modelado Predictivo}
[Describir aquí los modelos utilizados, estrategias de validación cruzada y selección de hiperparámetros.]

\section{Resultados}
[Incluir tablas de métricas, matrices de confusión y comparación con el baseline.]

\section{Discusión}
[Interpretar los resultados, analizar errores comunes y discutir hallazgos.]

\section{Conclusiones y Trabajo Futuro}
[Resumir los logros y proponer mejoras futuras.]

\end{document}
